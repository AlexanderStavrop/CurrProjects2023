\section*{Άσκηση 4}
\label{ex4}
\addcontentsline{toc}{section}{\nameref{ex4}}

\subsection*{Ερώτημα 1}
\label{ex4q1}
\addcontentsline{toc}{subsection}{\nameref{ex4q1}}

Σύμφωνα με τα δεδομένα κάθε πίνακα, είναι εμφανές πως και για το Α και για το Β η πιθανότητα να βρεθούν σε ένα οποιοδήποτε κανάλι ξεκινώντας από οποιοδήποτε κανάλι, είναι ίση $\frac{1}{3}$. Έτσι, είναι προφανές πως για να βρεθούν στο ίδιο κανάλι απαιτούνται $\frac{1}{\frac{1}{3}}$ βήματα (δηλαδή 3 βήματα) κατά μέσο όρο δηλαδή αναμένεται να υπάρχει "σύγκρουση" αν 3 βήματα κατά μέσο όρο.
\subsection*{Ερώτημα 2}
\label{ex4q2}
\addcontentsline{toc}{subsection}{\nameref{ex4q2}}