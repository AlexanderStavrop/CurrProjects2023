\section*{Άσκηση 3}
\label{ex3}
\addcontentsline{toc}{section}{\nameref{ex3}}

\subsection*{Ερώτημα 1}
\label{ex3q1}
\addcontentsline{toc}{subsection}{\nameref{ex3q1}}

Αναπαριστούμε τα states ως 3 αριθμοί, ο πρώτος είναι η τωρινή σελίδα και οι άλλοι 2 αριθμοί είναι οι σελίδες που είναι cached οπότε προκύπτουν 5 πιθανά states: (1,1,2) (2,1,2) (1,1,3) (3,1,3) (3,2,3) (2,2,3) και μέσω αυτών κατασκευάζεται ο πίνακας μετάβασης:


\noindent\\
Η λύση του προβλήματος προκύπτει λύνοντας τις παρακάτω σχέσεις:
\begin{align*}
    &\pi_{(3,2,3)} = \pi_{(2,2,3)}(1 - y) + \pi_{(1,1,2)}(1 - x) + \pi_{(2,1,2)}(1 - y) \\
    &\pi_{(2,2,3)} = \pi_{(3,2,3)} + \pi_{(3,1,3)} + \pi_{(1,1,3)}x \\ 
    &\pi_{(1,1,3)} = \pi_{(2,2,3)}y \\
    &\pi_{(3,1,3)} = \pi_{(1,1,3)}(1 - x)\\
    &\pi_{(2,1,2)} = \pi_{(1,1,2)}x \\
    & 1 = \pi_{(3,2,3)} + \pi_{(2,2,3)} + \pi_{(1,1,3)} + \pi_{(3,1,3)} + \pi_{(2,1,2)} + \pi_{(1,1,2)} \\
\end{align*}
 
οπότε λύνοντας τις παραπάνω σχέσει προκύπτουν οι παρακάτω τιμές:
\begin{align*}
    &\pi_{(2,1,2)} = 0 \\
    &\pi_{(1,1,2)} = 0 \\
    &\pi_{(2,2,3)} = \frac{1}{2 + y - xy}\\ 
    &\pi_{(3,1,3)} = \frac{(1-x)y}{2 + y - xy}\\
    &\pi_{(3,2,3)} = \frac{1-y}{2 + y - xy}\\
    &\pi_{(1,1,3)} = \frac{y}{2 + y - xy}\\
\end{align*}
 

\subsection*{Ερώτημα 2}
\label{ex3q2}
\addcontentsline{toc}{subsection}{\nameref{ex3q2}}


Για να βρεθεί ο αριθμός των request προς σελίδες που είναι ήδη cashed, έτσι η ζητούμενη τιμή είναι ίση με το άθροισμα όλων των request προς cashed σελίδες επί την πιθανότητα μετάβασης σε αυτή.