\section*{Άσκηση 4}
\label{ex4}
\addcontentsline{toc}{section}{\nameref{ex4}}

\subsection*{Ερώτημα 1}
\label{ex4q1}
\addcontentsline{toc}{subsection}{\nameref{ex4q1}}

Σύμφωνα με τα δεδομένα κάθε πίνακα, είναι εμφανές πως και για το Α και για το Β η πιθανότητα να βρεθούν σε ένα οποιοδήποτε κανάλι ξεκινώντας από οποιοδήποτε κανάλι, είναι ίση $\frac{1}{3}$. Έτσι, είναι προφανές πως για να βρεθούν στο ίδιο κανάλι απαιτούνται $\frac{1}{\frac{1}{3}}$ βήματα (δηλαδή 3 βήματα) κατά μέσο όρο δηλαδή αναμένεται να υπάρχει "σύγκρουση" αν 3 βήματα κατά μέσο όρο.

\subsection*{Ερώτημα 2}
\label{ex4q2}
\addcontentsline{toc}{subsection}{\nameref{ex4q2}}

Στο ερώτημα αυτό οι πίνακες δεν έχουν ίση πιθανότητα ($\frac{1}{3}$) και έτσι για την επίλυση το προβλήματος χρησιμοποιείται και πάλι Markov chain. Πιο συγκεκριμένα, η πιθανότητα να μεταδώσει σε νέο κανάλι κάθε μεταβλητή βασίζεται στο κανάλι που βρίσκεται τώρα και την πιθανότητα μετάβασης προς το νέο κανάλι.

\noindent\\
Είναι προφανές πως τα states (1,1), (2,2) (3,3) είναι absorbing καθώς όταν βρεθούν σε αυτά, τότε εκπέμπουν στο ίδιο κανάλι. Έτσι, ξεκινώντας από την κατάσταση ίδια αρχική κατάσταση (1,3) κατασκευάζεται ο παρακάτω πίνακας:
\begin{table}[h]
    \begin{tabular}{|c|c|c|c|c|c|c|c|c|c|}
    \hline
          & (1,1)     & (2,2)     & (3,3)     & (1,2)     & (1,3)     & (2,1)     & (2,3)     & (3,1)     & (3,2)     \\ \hline
    (1,1) & 1         & 0         & 0         & 0         & 0         & 0         & 0         & 0         & 0         \\ \hline
    (2,2) & 0         & 1         & 0         & 0         & 0         & 0         & 0         & 0         & 0         \\ \hline
    (3,3) & 0         & 0         & 1         & 0         & 0         & 0         & 0         & 0         & 0         \\ \hline
    (1,2) & 0.2 * 0.2 & 0.6 * 0.5 & 0.2* 0.3  & 0.2 * 0.5 & 0.2* 0.3  & 0.6 * 0.2 & 0.6 * 0.3 & 0.2 * 0.2 & 0.2 * 0.5 \\ \hline
    (1,3) & 0.2 * 0.3 & 0.6 *0.3  & 0.2 * 0.4 & 0.2 *0.3  & 0.2 * 0.4 & 0.6 * 0.3 & 0.6 * 0.4 & 0.2 * 0.3 & 0.2 * 0.3 \\ \hline
    (2,1) & 0.4 *0.4  & 0.3 * 0.3 & 0.3 * 0.3 & 0.4 * 0.3 & 0.4 * 0.3 & 0.3 * 0.4 & 0.3 * 0.3 & 0.3 *0.4  & 0.3 * 0.3 \\ \hline
    (2,3) & 0.4 * 0.3 & 0.3 * 0.3 & 0.3 * 0.4 & 0.4 * 0.3 & 0.4 * 0.4 & 0.3 * 0.3 & 0.3 * 0.4 & 0.3 * 0.3 & 0.3 * 0.3 \\ \hline
    (3,1) & 0.5 * 0.4 & 0.1 * 0.3 & 0.4 * 0.3 & 0.5 * 0.3 & 0.5 * 0.3 & 0.1 * 0.4 & 0.1 * 0.3 & 0.4 * 0.4 & 0.4 * 0.3 \\ \hline
    (3,2) & 0.5 * 0.2 & 0.1 * 0.5 & 0.4 * 0.3 & 0.5 * 0.5 & 0.5 * 0.3 & 0.1 * 0.2 & 0.1 * 0.3 & 0.4 * 0.2 & 0.4 * 0.5 \\ \hline
    \end{tabular}
\end{table}

Οπότε έχουμε:
\begin{table}[h]
    \begin{tabular}{lllllllllllll}
    \cline{1-9}
    \multicolumn{1}{|l|}{1}    & \multicolumn{1}{l|}{0}    & \multicolumn{1}{l|}{0}    & \multicolumn{1}{l|}{0}    & \multicolumn{1}{l|}{0}    & \multicolumn{1}{l|}{0}    & \multicolumn{1}{l|}{0}    & \multicolumn{1}{l|}{0}    & \multicolumn{1}{l|}{0}    &  &  &  &  \\ \cline{1-9}
    \multicolumn{1}{|l|}{0}    & \multicolumn{1}{l|}{1}    & \multicolumn{1}{l|}{0}    & \multicolumn{1}{l|}{0}    & \multicolumn{1}{l|}{0}    & \multicolumn{1}{l|}{0}    & \multicolumn{1}{l|}{0}    & \multicolumn{1}{l|}{0}    & \multicolumn{1}{l|}{0}    &  &  &  &  \\ \cline{1-9}
    \multicolumn{1}{|l|}{0}    & \multicolumn{1}{l|}{0}    & \multicolumn{1}{l|}{1}    & \multicolumn{1}{l|}{0}    & \multicolumn{1}{l|}{0}    & \multicolumn{1}{l|}{0}    & \multicolumn{1}{l|}{0}    & \multicolumn{1}{l|}{0}    & \multicolumn{1}{l|}{0}    &  &  &  &  \\ \cline{1-9}
    \multicolumn{1}{|l|}{0.04} & \multicolumn{1}{l|}{0.3}  & \multicolumn{1}{l|}{0.06} & \multicolumn{1}{l|}{0.1}  & \multicolumn{1}{l|}{0.06} & \multicolumn{1}{l|}{0.12} & \multicolumn{1}{l|}{0.18} & \multicolumn{1}{l|}{0.04} & \multicolumn{1}{l|}{0.1}  &  &  &  &  \\ \cline{1-9}
    \multicolumn{1}{|l|}{0.06} & \multicolumn{1}{l|}{0.18} & \multicolumn{1}{l|}{0.08} & \multicolumn{1}{l|}{0.06} & \multicolumn{1}{l|}{0.08} & \multicolumn{1}{l|}{0.18} & \multicolumn{1}{l|}{0.24} & \multicolumn{1}{l|}{0.06} & \multicolumn{1}{l|}{0.06} &  &  &  &  \\ \cline{1-9}
    \multicolumn{1}{|l|}{0.16} & \multicolumn{1}{l|}{0.09} & \multicolumn{1}{l|}{0.09} & \multicolumn{1}{l|}{0.12} & \multicolumn{1}{l|}{0.12} & \multicolumn{1}{l|}{0.12} & \multicolumn{1}{l|}{0.09} & \multicolumn{1}{l|}{0.12} & \multicolumn{1}{l|}{0.09} &  &  &  &  \\ \cline{1-9}
    \multicolumn{1}{|l|}{0.12} & \multicolumn{1}{l|}{0.09} & \multicolumn{1}{l|}{0.12} & \multicolumn{1}{l|}{0.12} & \multicolumn{1}{l|}{0.16} & \multicolumn{1}{l|}{0.09} & \multicolumn{1}{l|}{0.12} & \multicolumn{1}{l|}{0.09} & \multicolumn{1}{l|}{0.09} &  &  &  &  \\ \cline{1-9}
    \multicolumn{1}{|l|}{0.2}  & \multicolumn{1}{l|}{0.03} & \multicolumn{1}{l|}{0.12} & \multicolumn{1}{l|}{0.15} & \multicolumn{1}{l|}{0.15} & \multicolumn{1}{l|}{0.04} & \multicolumn{1}{l|}{0.03} & \multicolumn{1}{l|}{0.15} & \multicolumn{1}{l|}{0.12} &  &  &  &  \\ \cline{1-9}
    \multicolumn{1}{|l|}{0.1}  & \multicolumn{1}{l|}{0.05} & \multicolumn{1}{l|}{0.12} & \multicolumn{1}{l|}{0.25} & \multicolumn{1}{l|}{0.15} & \multicolumn{1}{l|}{0.02} & \multicolumn{1}{l|}{0.03} & \multicolumn{1}{l|}{0.08} & \multicolumn{1}{l|}{0.2}  &  &  &  &  \\ \cline{1-9}
                               &                           &                           &                           &                           &                           &                           &                           &                           &  &  &  & 
    \end{tabular}
\end{table}

όπου ο παραπάνω πίνακας ακολουθεί την μορφή  $P =\begin{bmatrix} Q & R \\
                                                                 0 & I_r   
                                                                
                                                \end{bmatrix} 
                                               $ με Q να είναι ο εξής πίνακας:

\begin{table}[h]
\begin{tabular}{llllllllll}
\cline{1-6}
\multicolumn{1}{|l|}{0.1}  & \multicolumn{1}{l|}{0.06} & \multicolumn{1}{l|}{0.12} & \multicolumn{1}{l|}{0.18} & \multicolumn{1}{l|}{0.04} & \multicolumn{1}{l|}{0.1}  &  &  &  &  \\ \cline{1-6}
\multicolumn{1}{|l|}{0.06} & \multicolumn{1}{l|}{0.08} & \multicolumn{1}{l|}{0.18} & \multicolumn{1}{l|}{0.24} & \multicolumn{1}{l|}{0.06} & \multicolumn{1}{l|}{0.06} &  &  &  &  \\ \cline{1-6}
\multicolumn{1}{|l|}{0.12} & \multicolumn{1}{l|}{0.12} & \multicolumn{1}{l|}{0.12} & \multicolumn{1}{l|}{0.09} & \multicolumn{1}{l|}{0.12} & \multicolumn{1}{l|}{0.09} &  &  &  &  \\ \cline{1-6}
\multicolumn{1}{|l|}{0.12} & \multicolumn{1}{l|}{0.16} & \multicolumn{1}{l|}{0.09} & \multicolumn{1}{l|}{0.12} & \multicolumn{1}{l|}{0.09} & \multicolumn{1}{l|}{0.09} &  &  &  &  \\ \cline{1-6}
\multicolumn{1}{|l|}{0.15} & \multicolumn{1}{l|}{0.15} & \multicolumn{1}{l|}{0.04} & \multicolumn{1}{l|}{0.03} & \multicolumn{1}{l|}{0.15} & \multicolumn{1}{l|}{0.12} &  &  &  &  \\ \cline{1-6}
\multicolumn{1}{|l|}{0.25} & \multicolumn{1}{l|}{0.15} & \multicolumn{1}{l|}{0.02} & \multicolumn{1}{l|}{0.03} & \multicolumn{1}{l|}{0.08} & \multicolumn{1}{l|}{0.2}  &  &  &  &  \\ \cline{1-6}
                        &                           &                           &                           &                           &                           &  &  &  & 
\end{tabular}
\end{table}


\clearpage
Τελίκα, η απάντηση του προβλήματος δίνεται από τον εξής πίνακα:
\begin{equation}
    N = (1 - Q)^{-1} = \begin{bmatrix} 1.335 & 0.273 & 0.294 & 0.394 & 0.195 & 0.294  \\
                                       0.319 & 1.325 & 0.381 & 0.483 & 0.232 & 0.273  \\
                                       0.374 & 0.349 & 1.312 & 0.326 & 0.293 & 0.301  \\
                                       0.375 & 0.392 & 0.290 & 1.369 & 0.262 & 0.302  \\
                                       0.404 & 0.372 & 0.226 & 0.264 & 1.328 & 0.332  \\
                                       0.541 & 0.394 & 0.229 & 0.299 & 0.256 & 1.445 
\end{bmatrix} 
\end{equation}

και προσθέτωντας τις τιμές της δεύτερης στήλης που αντιστοιχεί στο (1,3) προκύπτει η ζητούμενη απάντηση = 3,105  