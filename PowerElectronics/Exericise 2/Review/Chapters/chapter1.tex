\section*{Eισαγωγή}
\label{intro}
\addcontentsline{toc}{section}{\nameref{intro}}

\section{Περιγραφή λειτουργίας του συστήματος}
%\label{ex1}
%\addcontentsline{toc}{section}{\nameref{ex1}}

Οι ανορθωτές είναι κυκλώματα που μετατρέπουν το εναλλασσόμενο ρεύμα και τάση σε συνεχή είτε με έλεγχο (θυρίστορ) είτε χωρίς (δίοδοι). Η είσοδος μπορεί να είναι είτε μονοφασική είτε τριφασική, ενώ η έξοδος μπορεί να είναι είτε ελεγχόμενη είτε σταθερή. Στο σύστημα ενός ανορθωτή με ελεγχόμενη ανόρθωση πλήρους κύματος οι δίοδοι αντικαθίστανται από θυρίστορ. Τα θυρίστορ είναι μη γραμμικά στοιχεία που επιτρέπουν την διέλευση ρεύματος από το εσωτερικό
τους μόνο όταν δεχθούν παλμό έναυσης (γωνία α). Η αγωγή ρεύματος από το εσωτερικό των θυρίστορ διακόπτεται αν το ρεύμα γίνει αρνητικό ή μηδέν.

\section*{Ανορθωτής με ελενχόμενη ανόρθωση}

Οι ανορθωτές είναι κυκλώματα που μετατρέπουν το εναλλασσόμενο ρεύμα και τάση σε συνεχή είτε με έλεγχο (θυρίστορ) είτε χωρίς (δίοδοι). Η είσοδος μπορεί να είναι είτε μονοφασική είτε τριφασική, ενώ η έξοδος μπορεί να είναι είτε ελεγχόμενη είτε σταθερή. Στο σύστημα ενός ανορθωτή με ελεγχόμενη ανόρθωση πλήρους κύματος οι δίοδοι αντικαθίστανται από θυρίστορ. Τα θυρίστορ είναι μη γραμμικά στοιχεία που επιτρέπουν την διέλευση ρεύματος από το εσωτερικό
τους μόνο όταν δεχθούν παλμό έναυσης (γωνία α). Η αγωγή ρεύματος από το εσωτερικό των θυρίστορ διακόπτεται αν το ρεύμα γίνει αρνητικό ή μηδέν.


\section*{1.1 Μονοφασικός ανορθωτής με ελενχόμενη ανόρθωση, πλήρους κύματος με R-L φορτίο}

Οι ανορθωτές ελεγχόμενης ανόρθωσης όπως και οι αντίστοιχοι μη ελεγχόμενης διακρίνονται σε επιμέρους φάσεις λειτουργίας. Για μονοφασικό ανορθωτή με α=0 (κύκλωμα με θυρίστορ αλλά λειτουργία αντίστοιχη με διόδους). Κατά την Φ1 άγουν τα T1, T2, ενώ κατά την Φ2 άγουν τα T3, T4. Στην Φ1 το ρεύμα φορτίου i0 συμπίπτει με το ρεύμα εισόδου Is, ενώ στην Φ2 είναι αντίθετο με το is.
Η λειτουργία των μονοφασικών ανορθωτών χωρίζεται ανάλογα με τις τιμές που δέχεται το ρεύμα σε δύο κατηγορίες:


\subsection*{Ασυνεχής Λειτουργία}

Στην ασυνεχή λειτουργία υπάρχουν κάποιες χρονικές περίοδοι στις οποίες το ρέυμα μηδενίζεται. Αυτό οφείλεται κυρίως στη μικρή αυτεπαγωγή του φορτίου, η οποία δεν καταφέρνει να σταθεροποιήσει το ρεύμα σε τέτοιο βαθμό που να μην προλάβει να μηδενιστεί. Στο σημείο μηδενισμού του ρεύματος ορίζεται η γωνία σβέσης β. Η οριακή συνθήκη συνεχούς ασυνεχούς λειτουργίας είναι η β = π + α.
Όσον αφορά την τάση στην έξοδο, εκεί που μηδενίζει το ρεύμα μηδενίζει και αυτή με αποτέλεσμα τη δημιουργία ενός σκαλοπατιού, έως ότου τα θυρίστορ δεχθούν εκ νέου παλμό έναυσης και αφήσουν το ρεύμα να διέλθει από αυτά. Στο κύκλωμα με R-L στοιχείο στην έξοδο, το ρέυμα λόγω της αυτεπαγωγής καθυστερεί σε σχέση με την τάση με αποτέλεσμα η τάση να λαμβάνει και αρνητικές τιμές.


\subsection*{Συνεχής Λειτουργία}

Στη συνεχή λειτουργία συμβαίνουν τα ακριβώς αντίθετα. Συγκεκριμένα, το ρεύμα δεν μηδενίζει, που έχει ως αποτέλεσμα η τάση να μη μηδενίζει. Η δεύτερη ακολουθεί ένα τμήμα της ημιτονοειδούς μορφής της τάσης εισόδου. Λόγω του μη αρνητικού ή μηδενικού ρεύματος συμπεραίνεται πως η γωνία σβέσης παίρνει τιμή μεγαλύτερη της οριακής συνθήκης συνεχούς και ασυνεχούς λειτουργίας (β > π +α). Αυτό που ισχύει τόσο στη συνεχή όσο και στην ασυνεχή λειτουργία είναι πως μεγαλύτερη αυτεπαγωγή συνεπάγεται καλύτερη σταθεροποίηση της κυματομορφής του ρεύματος.

\newpage
\section*{1.2 Τριφασικός ανορθωτής με ελενχόμενη ανόρθωση και RL φορτίο}

Στο μονοφασικό ανορθωτή ελεγχόμενης ανόρθωσης τα θυρίστορ ήταν 4 εκ των οποίων 2 ήγαγαν σε κάθε φάση λειτουργίας. Στον τριφασικό ανορθωτή προστίθενται άλλα 2 θυρίστορ. Σε κάθε περιοχή λειτουργίας πάλι άγουν ακριβώς 2 θυρίστορ. Για την ανακάλυψη του κατάλληλου συνδυασμού από θυρίστορ εξετάζουμε την κοινή κάθοδο και την κοινή άνοδο του πάνω και του κάτω συμπλέγματος από θυρίστορ αντίστοιχα. Συγκεκριμένα, εξετάζουμε τις στιγμιαίες τιμές των φασικών τάσεων. Στο κάτω σύμπλεγμα θα άγει αυτή που έχει την πιο αρνητική κάθοδο, ενώ
στο πάνω σύμπλεγμα αυτή που έχει την πιο θετική άνοδο. Φυσικά, τα θυρίστορ άγουν αν και μόνο αν λάβουν παλμό έναυσης στην είσοδό τους. Η εύρεση του κατάλληλου ζευγαριού απαιτεί ιδιαίτερη προσοχή καθώς οι βρόχοι των θυρίστορ
αλλάζουν συνέχως. Συγκεκριμένα σε μια περίοδο αλλάζουν 6 φορές. Αυτό σημαίνει πως η συχνότητα του ανορθωμένου σήματος εξόδου εξαπλασιάζεται, με αποτέλεσμα το σήμα να γίνεται πιο σταθερό, γεγονός που θα φανεί και στις κυματομορφές του πειράματος.
