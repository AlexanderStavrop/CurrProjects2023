\section*{Ερώτηση 4 - THD, DPF, Στιγμιαία Ισχύς, S, P, PF}
\label{ex4}
\addcontentsline{toc}{section}{\nameref{ex4}}

\subsection*{Συντελεστή ολικής αρμονικής παραμόρφωσης - THD}
\label{ex4q1}
\addcontentsline{toc}{subsection}{\nameref{ex4q1}}

Ο συντελεστής συνολικής αρμονικής παραμόρφωσης (THD) είναι το μέτρο παραμόρφωσης ενός σήματος που προκαλείται από την παρουσία αρμονικών. Εκφράζεται ως ποσοστό και αντιπροσωπεύει τον λόγο του αθροίσματος των δυνάμεων όλων των αρμονικών εκτός της βασικής προς την ισχύ της βασικής αρμονικής, είναι δηλαδή ένα μέτρο του πόσο συμβάλλουν οι αρμονικές συχνότητες στη συνολική παραμόρφωση του σήματος. 

\noindent\\
Ο τύπος για τον υπολογισμό του THD για ρεύμα και τάση είναι ο εξής: 
\begin{equation}
    THD = \left\{ \begin{array}{l}
                    100\cdot\frac{\sqrt{\sum _{n\neq1}^{ }I^2_{Sh,rms}\:}}{I_{S1,rms}}
                    \\
                    \\
                    100\cdot\frac{\sqrt{\sum _{n\neq1}^{ }v^2_{Sh,rms}\:}}{v_{S1,rms}}
                \end{array}
          \right. \label{THD}
\quad                                                                              
\end{equation}

\noindent\\
Αντικαθιστώντας τα σήματα τάσης και ρεύματος στην σχέση (\ref{THD}) προκύπτουν τα εξής αποτελέσματα:	
\begin{itemize}
    \item $THD_{current} = 1.9867\%$
    \item $THD_{voltage1} = 145.6403\%$
    \item $THD_{voltage2} = 70.3831\%$
    \item $THD_{voltage3} = 5.1766\%$
\end{itemize}

\noindent
Οταν το THD ενός σήματος τείνει στο 0 σημαίνει πως υπάρχει ελάχιστη αλλοίωση, δηλαδή έχει λίγες αρμονικές. Ετσι, όπως είναι αναμενόμενο τα \textit{voltage1} και \textit{voltage2} εμφανίζουν μεγάλη αλλοίωση $145.6403\%$ και $70.3831\%$ αντιστοίχως λόγω μεγάλου αριθμού αρμονικών που εμφανίζονται σε αυτά μετά από έλειπες φιλτράρισμα ενώ τα \textit{voltage3} και \textit{current} έχουν μικρή αλλοίωση $5.1766\%$ και $1.9867\%$ αντιστοίχως λόγω επαρκούς φιλτραρίσματος το οποίο έχει ως αποτέλεσμα την μείωση των αρμονικών.
	
\subsection*{Ο συντελεστής ισχύος μετατόπισης - DPF}
\label{ex4q2}
\addcontentsline{toc}{subsection}{\nameref{ex4q2}}

 Ο συντελεστής ισχύος μετατόπισης (DPF) είναι μια παράμετρος που περιγράφει τη φασική σχέση μεταξύ της τάσης και του ρεύματος σε ένα κύκλωμα και δείχνει πόσο αποτελεσματικά χρησιμοποιείται η παρεχόμενη τάση για την παραγωγή χρήσιμου έργου. Οταν το DPF ισούται με 1 υποδηλώνει πως οι κυματομορφές τάσης και ρεύματος είναι απόλυτα σε φάση και έτσι όλη η παρεχόμενη ισχύς χρησιμοποιείται για χρήσιμο έργο, ενώ όταν η τιμή του είναι μικρότερη απο 1 αυτό σημαίνει πως υπάρχουν απώλειες στο κύκλωμα. 
 
\noindent\\
Ο τύπος απο τον οποίο υπολογισμού του DPF είναι ο εξής:

\begin{equation}
    DPF = cos(\phi_{voltage}-\phi_{current}) \label{DPF}
\end{equation}

\noindent \\
Εφαρμόζοντας τις τρις αντίστοιχες γωνίες των τάσεων στη σχέση (\ref{DPF}) προκύπτουν τα εξής αποτελέσματα:
	 
\begin{itemize}
    \item $DPF_{Voltage1} = 0.3093$
    \item $DPF_{Voltage2} = 0.3827$
    \item $DPF_{Voltage3} = 0.4540$
\end{itemize}
\noindent
Παρατηρώντας τα αποτελέσματα είναι εμφανές πως όσο λιγότερες αρμονικές έχει μια τάση τόσο μεγαλύτερο είναι το DPF. Αυτό συμβαίνει καθώς οι αρμονικές είναι πρόσθετες ημιτονοειδείς κυματομορφές με συχνότητες που είναι ακέραια πολλαπλάσια της θεμελιώδους συχνότητας του κυκλώματος. Όταν υπάρχουν αρμονικές στο ρεύμα ή στην τάση προκαλείται  παραμόρφωση της κυματομορφής η οποία οδηγεί σε μετατόπιση φάσης μεταξύ των κυματομορφών τάσης και ρεύματος προκαλώντας έτσι μείωση του DPF.
	
\subsection*{Στιγμιαία ισχύς}
\label{ex4q3}
\addcontentsline{toc}{subsection}{\nameref{ex4q3}}
Η στιγμιαία ισχύς υπολογίζεται πολλαπλασιάζοντας την τάση με το ρεύμα για μία δεδομένη χρονική στιγμή:
\begin{equation}
    P(t)=v(t) \cdot i(t)
\end{equation}
Οι γραφικές αναπαραστάσεις των στιγμιαίων ισχύων για κάθε τάση φαίνονται στο figure (\ref{instantaneous}):

\begin{figure}[h]
    \centering
    \includegraphics[width=1\textwidth]{Images/Q4_voltages.eps}
    \caption{Στιγμιαίες Ισχύες}
    \label{instantaneous}
\end{figure}

\noindent
Παρατηρώντας το figure (\ref{instantaneous}) είναι εμφανές πως όσο μικρότερος ο αριθμός των αρμονικών τόσο πιο ημιτονοειδές είναι η γραφική της στιγμιαίας ισχύος. Πιο συγκεκριμένα, όπως φαίνεται στην γραφική αναπαράσταση της στιγμιαίας ισχύος του σήματος \textit{voltage1}, το οποίο έχει τις περισσότερες αρμονικές, η αλλίωση είναι η πιο έντονη και όσο μειώνονται οι αρμονικές για τα σήματα \textit{voltage2} και \textit{voltage3} αντίστοιχα, οι γραφικές ισχύος είναι όλο και λιγότερο αλλοιωμένες.

\subsection*{Φαινόμενη ισχύς - S}
\label{ex4q4}
\addcontentsline{toc}{subsection}{\nameref{ex4q4}}
Η φαινόμενη ισχύς είναι η ισχύς που καταναλώνεται ή παράγεται από ένα ηλεκτρικό κύκλωμα, χωρίς να λαμβάνεται υπόψην η γωνία φάσης μεταξύ τάσης και ρεύματος και υπολογίζεται ως εξής:
\begin{equation}
    S = V_{rms} \cdot I_{rms} \text{ όπου oι rms τιμές υπολογίζονται ως εξής: } \left\{ \begin{array}{l}
                                                                                            V_{rms}=\sqrt{V_0^2+\sum_{h=1}^{\infty }V_h^2\:}\\
                                                                                            \\
                                                                                            I_{rms}=\sqrt{I_0^2+\sum_{h=1}^{\infty }I_h^2\:}
                                                                                        \end{array}
                                                                                \right. 
\quad                                                                              
\end{equation} 

\noindent\\
Αντικαθιστώντας τις τάσεις προκύπτουν τα εξής αποτελέσματα:
\begin{itemize}
    \item $S_{Voltage1} = 4284.7781VA$
    \item $S_{Voltage2} = 2962.8030VA$
    \item $S_{Voltage3} = 2418.6192VA$
\end{itemize}
\noindent
Παρατηρείτε πως οσο περισσότερες αρμονικές έχει το σήμα τάσης, τόσο μικρότερη είναι η αντίστοιχη ενεργός ισχύς. Αυτό συμβαίνει καθώς οι αρμονικές αυξάνουν την άεργο ισχύ το οποίο συνεπάγεται με αύξηση της φαινόμενης ισχύς καθώς ισούται με άθροισμα της άεργου και της ενεργού.


\subsection*{Ενεργός Ισχύ - P}
\label{ex4q5}
\addcontentsline{toc}{subsection}{\nameref{ex4q5}}
Η Ενεργός Ισχύ είναι η ισχύς που μετατρέπεται πραγματικά σε χρήσιμη εργασία από ένα ηλεκτρικό κύκλωμα και υπολογίζεται ως το άρθοισμα όλων των στιγμιάιων ισχύων διαρεμένο με το μήκος του συνολικού χρόνου. 
\begin{equation}
    P_{a} =\frac{\sum_{k=0}^{} P_{i}(k)}{L} \label{P}
\end{equation}

\noindent\\
Αντικαθιστώντας καί τις τρείς στιγμαιές ισχύες στην σχέση (\ref{P}) προκύπτουν τα εξής αποτελέσματα:
\begin{itemize}
    \item $P_{Voltage1} = 751.2989W$
    \item $P_{Voltage2} = 933.3890W$
    \item $P_{Voltage3} = 1096.112W$
\end{itemize}

\noindent\\
Παρατηρώντας τα αποτελέσματα είναι εμφανές πως όσο περισσότερες αρμονικές υπάρχουν στο σήμα της τάσης τόσο μικρότερη είναι η αντίστοιχη ενεργός ισχύς.


\subsection*{Συντελεστή Ισχύος - PF}
\label{ex4q6}
\addcontentsline{toc}{subsection}{\nameref{ex4q6}}
Ο συντελεστής ισχύος (PF) ορίζεται ως ο λόγος της πραγματικής ισχύος ($P_{a}$) προς τη φαινόμενη ισχύ (S) σε ένα κύκλωμα. Η πραγματική ισχύς αντιπροσωπεύει την ισχύ που καταναλώνεται πραγματικά από το κύκλωμα και χρησιμοποιείται για χρήσιμο έργο, ενώ η φαινόμενη ισχύς αντιπροσωπεύει τη συνολική ισχύ που παρέχεται στο κύκλωμα, συμπεριλαμβανομένης της άεργου ισχύος που χρησιμοποιείται για μη χρήσιμο έργο.

\noindent\\
Η διαφορά μεταξύ του PF με τον  DPF είναι πως ο PF λαμβάνει υπόψην όλη την ισχύ που παρέχεται στο κύκλωμα, συμπεριλαμβανομένης της άεργου ισχύος που χρησιμοποιείται για μη χρήσιμες εργασίες, ενώ ο DPF λαμβάνει υπόψη μόνο τη φασική σχέση μεταξύ των κυματομορφών τάσης και ρεύματος.

\noindent\\
O PF υπολογίζεται ως εξής:
\begin{equation} 
    PF=\frac{P_{a}}{S} \label{PF}
\end{equation} 

\noindent\\
Αντικαθιστώντας τις τάσεις στην σχέση (\ref{PF}) προκύπτουν τα εξής αποτελέσματα:
\begin{itemize}
    \item $PF_{Voltage1} = 0.1753$
    \item $PF_{Voltage2} = 0.3150$
    \item $PF_{Voltage3} = 0.4531$
\end{itemize}

\noindent\\
Για τον ίδιο λόγο με το προηγούμενο ερώτημα, οσο λιγότερες είναι οι αρμονικές τόσο μεγαλύτερος είναι ο PF. Επιπλέον, αξίζει να σημειωθεί πως όπως ήταν αναμενόμενο,  ο PF είναι μικρότερος απο τον αντίστοιχο DPF εφόσον όπως προαναφέρθηκε, ο PF λαμβάνει υπόψη καί την άεργη ισχύ σε αντίθεση με τον DPF.