\section*{Ερώτηση 5 - Υπολογισμός rms τιμής}
\label{ex5}
\addcontentsline{toc}{section}{\nameref{ex5}}

\section*{Υπολογισμός rms τιμής με βάση τον ορισμό}
\label{ex5q1}
\addcontentsline{toc}{subsection}{\nameref{ex5q1}}


Για τον υπολογισμό μίας rms τιμής ενός σήματος συνεχούς χρόνου αξιοποιείται ο εξής τύπος:
\begin{equation}
    X_{rms} = \sqrt{\frac{1}{T} \cdot \int_0^T x^2(t) dt} \label{X_rms_C}
\end{equation}

\noindent\\
Tο σήμα \textit{voltage2} είναι διακριτού χρόνου και έτσι για την εύρεση της rms τιμής δεν μπορεί να χρησιμοποιηθεί η σχέση (\ref{X_rms_C}) αλλά απαιτείται η χρήση προσεγγιστικού τύπου. Πιο συγκεκριμένα, για τον υπολογισμό της rms τιμής αξιοποιήθηκαν οι τιμές του σήματος \textit{voltage2} και του διανύσματος του χρόνου (\textit{time}) σε διάστημα μίας μόνο περιόδου, ώστε να είναι σίγουρο πως η rms τιμή υπολογίζεται σε ακέραιο πολλαπλάσιο μίας περιόδου του σήματος. 

\noindent\\
Ο τύπος υπολογισμού της τιμής rms ορίζεται ως το άθροισμα όλων των σημείων μίας περιόδου προς το πλήθως αυτων των σημείων, δηλαδή η μέση τιμή. Το  πυλίκό αυτών των όρων τοποθετείται σε ρίζα ώστε να προκύψει ο τελικός τύπος της rms όπως φαίνεται στην σχέση (\ref{X_rms_th}):
\begin{equation}
    X_{rms_{th}} = \sqrt{\frac{1}{L_{t\_period}} \cdot \sum_{n=0}^{L_{t\_period} - 1} x(n)^2} \label{X_rms_th}
\end{equation}

όπου $L_{t\_period}$ το μήκος μίας περιόδου.

\noindent\\
Αντικαθιστώντας το σήμα \textit{voltage2} στην σχέση \ref{X_rms_exp} το αποτέλεσμα προκύπτει ίσο με: 69.1438

\section*{Υπολογισμός rms τιμής μέσω του φασματικού περιεχομένου έως 700Hz}
\label{ex5q2}
\addcontentsline{toc}{subsection}{\nameref{ex5q2}}

Για τον υπολογισμό της rms τιμής μέσω του φασματικού περιεχομένου αρχικά έγινε περιορισμός του σήματος έως τα 700Hz και χρησιμοποιήθηκε ο εξής τύπος:
\begin{equation}
    X_{rms_{exp}} = \sqrt{\sum_{k=0}^{L_{700}} X(k)^2} \label{X_rms_exp}
\end{equation}

όπου $L_{700}$, το μήκος του σήματος έως τα 700Hz.

\noindent\\
Αντικαθιστώντας το σήμα \textit{voltage2} στην σχέση (\ref{X_rms_exp}) και το αποτέλεσμα είναι ίσο με: 69.1146

\section*{Σύγκριση αποτελεσμάτων}
\label{ex5q3}
\addcontentsline{toc}{subsection}{\nameref{ex5q3}}

Συγκρίνοντας τις παραπάνω τιμές παρατηρείται πως η θεωρητική αποκλίνει ελάχιστα ($\approx 0.04\%$). Οι δύο τιμές αναμένεται να ισούται μεταξύ τους, ωστόσο αυτή η πολύ μιρκή διαφορά οφείλεται στο γεγονώς πως για τον υπολογισμό της rms μέσω του φασματικού περιεχομένου αξιοποιήθηκαν συχνότητες ως και τα 700Hz.

\noindent\\
Η επιλογή περιορισμού του σήματος, φαίνεται να παράγει αρκετά καλή προσέγγιση σε σχέση με την θεωρητική τιμή και προσδίδει το προτέρημα πως απαιτείται λιγότερος χώρος για την παραστάση του σήματος.
